\documentclass[11pt]{article}
\renewcommand{\arraystretch}{1.5} % Default value: 1
\usepackage{sectsty}
\allsectionsfont{\color{blue}\fontfamily{lmss}\selectfont}
\usepackage{fontspec}
\setmainfont{XCharter}

\usepackage{listings}

\lstset{
basicstyle=\small\ttfamily,
tabsize=8,
columns=flexible,
breaklines=true,
frame=tb,
rulecolor=\color[rgb]{0.8,0.8,0.7},
backgroundcolor=\color[rgb]{1,1,0.91},
postbreak=\raisebox{0ex}[0ex][0ex]{\ensuremath{\color{red}\hookrightarrow\space}}
}
\usepackage{fontawesome}


\usepackage{mdframed}
\newmdenv[
  backgroundcolor=gray,
  fontcolor=white,
  nobreak=true,
]{terminalinput}



\usepackage{parskip}


    \usepackage[breakable]{tcolorbox}
    \usepackage{parskip} % Stop auto-indenting (to mimic markdown behaviour)

    \usepackage{iftex}
    \ifPDFTeX
    	\usepackage[T1]{fontenc}
    	\usepackage{mathpazo}
    \else
    	\usepackage{fontspec}
    \fi

    % Basic figure setup, for now with no caption control since it's done
    % automatically by Pandoc (which extracts ![](path) syntax from Markdown).
    \usepackage{graphicx}
    % Maintain compatibility with old templates. Remove in nbconvert 6.0
    \let\Oldincludegraphics\includegraphics
    % Ensure that by default, figures have no caption (until we provide a
    % proper Figure object with a Caption API and a way to capture that
    % in the conversion process - todo).
    \usepackage{caption}
    \DeclareCaptionFormat{nocaption}{}
    \captionsetup{labelformat=nolabel, textfont=bf}

    \usepackage{float}
    \floatplacement{figure}{H} % forces figures to be placed at the correct location
    \usepackage{xcolor} % Allow colors to be defined
    \usepackage{enumerate} % Needed for markdown enumerations to work
    \usepackage{geometry} % Used to adjust the document margins
    \usepackage{amsmath} % Equations
    \usepackage{amssymb} % Equations
    \usepackage{textcomp} % defines textquotesingle
    % Hack from http://tex.stackexchange.com/a/47451/13684:
    \AtBeginDocument{%
        \def\PYZsq{\textquotesingle}% Upright quotes in Pygmentized code
    }
    \usepackage{upquote} % Upright quotes for verbatim code
    \usepackage{eurosym} % defines \euro
    \usepackage[mathletters]{ucs} % Extended unicode (utf-8) support
    \usepackage{fancyvrb} % verbatim replacement that allows latex
    \usepackage{grffile} % extends the file name processing of package graphics
                         % to support a larger range
    \makeatletter % fix for old versions of grffile with XeLaTeX
    \@ifpackagelater{grffile}{2019/11/01}
    {
      % Do nothing on new versions
    }
    {
      \def\Gread@@xetex#1{%
        \IfFileExists{"\Gin@base".bb}%
        {\Gread@eps{\Gin@base.bb}}%
        {\Gread@@xetex@aux#1}%
      }
    }
    \makeatother
    \usepackage[Export]{adjustbox} % Used to constrain images to a maximum size
    \adjustboxset{max size={0.9\linewidth}{0.9\paperheight}}

    % The hyperref package gives us a pdf with properly built
    % internal navigation ('pdf bookmarks' for the table of contents,
    % internal cross-reference links, web links for URLs, etc.)
    \usepackage{hyperref}
    % The default LaTeX title has an obnoxious amount of whitespace. By default,
    % titling removes some of it. It also provides customization options.
    \usepackage{titling}
    \usepackage{longtable} % longtable support required by pandoc >1.10
    \usepackage{booktabs}  % table support for pandoc > 1.12.2
    \usepackage[inline]{enumitem} % IRkernel/repr support (it uses the enumerate* environment)
    \usepackage[normalem]{ulem} % ulem is needed to support strikethroughs (\sout)
                                % normalem makes italics be italics, not underlines
    \usepackage{mathrsfs}



    % Colors for the hyperref package
    \definecolor{urlcolor}{rgb}{0,.145,.698}
    \definecolor{linkcolor}{rgb}{.71,0.21,0.01}
    \definecolor{citecolor}{rgb}{.12,.54,.11}

    % ANSI colors
    \definecolor{ansi-black}{HTML}{3E424D}
    \definecolor{ansi-black-intense}{HTML}{282C36}
    \definecolor{ansi-red}{HTML}{E75C58}
    \definecolor{ansi-red-intense}{HTML}{B22B31}
    \definecolor{ansi-green}{HTML}{00A250}
    \definecolor{ansi-green-intense}{HTML}{007427}
    \definecolor{ansi-yellow}{HTML}{DDB62B}
    \definecolor{ansi-yellow-intense}{HTML}{B27D12}
    \definecolor{ansi-blue}{HTML}{208FFB}
    \definecolor{ansi-blue-intense}{HTML}{0065CA}
    \definecolor{ansi-magenta}{HTML}{D160C4}
    \definecolor{ansi-magenta-intense}{HTML}{A03196}
    \definecolor{ansi-cyan}{HTML}{60C6C8}
    \definecolor{ansi-cyan-intense}{HTML}{258F8F}
    \definecolor{ansi-white}{HTML}{C5C1B4}
    \definecolor{ansi-white-intense}{HTML}{A1A6B2}
    \definecolor{ansi-default-inverse-fg}{HTML}{FFFFFF}
    \definecolor{ansi-default-inverse-bg}{HTML}{000000}

    % common color for the border for error outputs.
    \definecolor{outerrorbackground}{HTML}{FFDFDF}

    % commands and environments needed by pandoc snippets
    % extracted from the output of `pandoc -s`
    \providecommand{\tightlist}{%
      \setlength{\itemsep}{0pt}\setlength{\parskip}{0pt}}
    \DefineVerbatimEnvironment{Highlighting}{Verbatim}{commandchars=\\\{\}}
    % Add ',fontsize=\small' for more characters per line
    \newenvironment{Shaded}{}{}
    \newcommand{\KeywordTok}[1]{\textcolor[rgb]{0.00,0.44,0.13}{\textbf{{#1}}}}
    \newcommand{\DataTypeTok}[1]{\textcolor[rgb]{0.56,0.13,0.00}{{#1}}}
    \newcommand{\DecValTok}[1]{\textcolor[rgb]{0.25,0.63,0.44}{{#1}}}
    \newcommand{\BaseNTok}[1]{\textcolor[rgb]{0.25,0.63,0.44}{{#1}}}
    \newcommand{\FloatTok}[1]{\textcolor[rgb]{0.25,0.63,0.44}{{#1}}}
    \newcommand{\CharTok}[1]{\textcolor[rgb]{0.25,0.44,0.63}{{#1}}}
    \newcommand{\StringTok}[1]{\textcolor[rgb]{0.25,0.44,0.63}{{#1}}}
    \newcommand{\CommentTok}[1]{\textcolor[rgb]{0.38,0.63,0.69}{\textit{{#1}}}}
    \newcommand{\OtherTok}[1]{\textcolor[rgb]{0.00,0.44,0.13}{{#1}}}
    \newcommand{\AlertTok}[1]{\textcolor[rgb]{1.00,0.00,0.00}{\textbf{{#1}}}}
    \newcommand{\FunctionTok}[1]{\textcolor[rgb]{0.02,0.16,0.49}{{#1}}}
    \newcommand{\RegionMarkerTok}[1]{{#1}}
    \newcommand{\ErrorTok}[1]{\textcolor[rgb]{1.00,0.00,0.00}{\textbf{{#1}}}}
    \newcommand{\NormalTok}[1]{{#1}}

    % Additional commands for more recent versions of Pandoc
    \newcommand{\ConstantTok}[1]{\textcolor[rgb]{0.53,0.00,0.00}{{#1}}}
    \newcommand{\SpecialCharTok}[1]{\textcolor[rgb]{0.25,0.44,0.63}{{#1}}}
    \newcommand{\VerbatimStringTok}[1]{\textcolor[rgb]{0.25,0.44,0.63}{{#1}}}
    \newcommand{\SpecialStringTok}[1]{\textcolor[rgb]{0.73,0.40,0.53}{{#1}}}
    \newcommand{\ImportTok}[1]{{#1}}
    \newcommand{\DocumentationTok}[1]{\textcolor[rgb]{0.73,0.13,0.13}{\textit{{#1}}}}
    \newcommand{\AnnotationTok}[1]{\textcolor[rgb]{0.38,0.63,0.69}{\textbf{\textit{{#1}}}}}
    \newcommand{\CommentVarTok}[1]{\textcolor[rgb]{0.38,0.63,0.69}{\textbf{\textit{{#1}}}}}
    \newcommand{\VariableTok}[1]{\textcolor[rgb]{0.10,0.09,0.49}{{#1}}}
    \newcommand{\ControlFlowTok}[1]{\textcolor[rgb]{0.00,0.44,0.13}{\textbf{{#1}}}}
    \newcommand{\OperatorTok}[1]{\textcolor[rgb]{0.40,0.40,0.40}{{#1}}}
    \newcommand{\BuiltInTok}[1]{{#1}}
    \newcommand{\ExtensionTok}[1]{{#1}}
    \newcommand{\PreprocessorTok}[1]{\textcolor[rgb]{0.74,0.48,0.00}{{#1}}}
    \newcommand{\AttributeTok}[1]{\textcolor[rgb]{0.49,0.56,0.16}{{#1}}}
    \newcommand{\InformationTok}[1]{\textcolor[rgb]{0.38,0.63,0.69}{\textbf{\textit{{#1}}}}}
    \newcommand{\WarningTok}[1]{\textcolor[rgb]{0.38,0.63,0.69}{\textbf{\textit{{#1}}}}}


    % Define a nice break command that doesn't care if a line doesn't already
    % exist.
    \def\br{\hspace*{\fill} \\* }
    % Math Jax compatibility definitions
    \def\gt{>}
    \def\lt{<}
    \let\Oldtex\TeX
    \let\Oldlatex\LaTeX
    \renewcommand{\TeX}{\textrm{\Oldtex}}
    \renewcommand{\LaTeX}{\textrm{\Oldlatex}}
    % Document parameters
    % Document title
    \title{Answers}





% Pygments definitions
\makeatletter
\def\PY@reset{\let\PY@it=\relax \let\PY@bf=\relax%
    \let\PY@ul=\relax \let\PY@tc=\relax%
    \let\PY@bc=\relax \let\PY@ff=\relax}
\def\PY@tok#1{\csname PY@tok@#1\endcsname}
\def\PY@toks#1+{\ifx\relax#1\empty\else%
    \PY@tok{#1}\expandafter\PY@toks\fi}
\def\PY@do#1{\PY@bc{\PY@tc{\PY@ul{%
    \PY@it{\PY@bf{\PY@ff{#1}}}}}}}
\def\PY#1#2{\PY@reset\PY@toks#1+\relax+\PY@do{#2}}

\expandafter\def\csname PY@tok@w\endcsname{\def\PY@tc##1{\textcolor[rgb]{0.73,0.73,0.73}{##1}}}
\expandafter\def\csname PY@tok@c\endcsname{\let\PY@it=\textit\def\PY@tc##1{\textcolor[rgb]{0.25,0.50,0.50}{##1}}}
\expandafter\def\csname PY@tok@cp\endcsname{\def\PY@tc##1{\textcolor[rgb]{0.74,0.48,0.00}{##1}}}
\expandafter\def\csname PY@tok@k\endcsname{\let\PY@bf=\textbf\def\PY@tc##1{\textcolor[rgb]{0.00,0.50,0.00}{##1}}}
\expandafter\def\csname PY@tok@kp\endcsname{\def\PY@tc##1{\textcolor[rgb]{0.00,0.50,0.00}{##1}}}
\expandafter\def\csname PY@tok@kt\endcsname{\def\PY@tc##1{\textcolor[rgb]{0.69,0.00,0.25}{##1}}}
\expandafter\def\csname PY@tok@o\endcsname{\def\PY@tc##1{\textcolor[rgb]{0.40,0.40,0.40}{##1}}}
\expandafter\def\csname PY@tok@ow\endcsname{\let\PY@bf=\textbf\def\PY@tc##1{\textcolor[rgb]{0.67,0.13,1.00}{##1}}}
\expandafter\def\csname PY@tok@nb\endcsname{\def\PY@tc##1{\textcolor[rgb]{0.00,0.50,0.00}{##1}}}
\expandafter\def\csname PY@tok@nf\endcsname{\def\PY@tc##1{\textcolor[rgb]{0.00,0.00,1.00}{##1}}}
\expandafter\def\csname PY@tok@nc\endcsname{\let\PY@bf=\textbf\def\PY@tc##1{\textcolor[rgb]{0.00,0.00,1.00}{##1}}}
\expandafter\def\csname PY@tok@nn\endcsname{\let\PY@bf=\textbf\def\PY@tc##1{\textcolor[rgb]{0.00,0.00,1.00}{##1}}}
\expandafter\def\csname PY@tok@ne\endcsname{\let\PY@bf=\textbf\def\PY@tc##1{\textcolor[rgb]{0.82,0.25,0.23}{##1}}}
\expandafter\def\csname PY@tok@nv\endcsname{\def\PY@tc##1{\textcolor[rgb]{0.10,0.09,0.49}{##1}}}
\expandafter\def\csname PY@tok@no\endcsname{\def\PY@tc##1{\textcolor[rgb]{0.53,0.00,0.00}{##1}}}
\expandafter\def\csname PY@tok@nl\endcsname{\def\PY@tc##1{\textcolor[rgb]{0.63,0.63,0.00}{##1}}}
\expandafter\def\csname PY@tok@ni\endcsname{\let\PY@bf=\textbf\def\PY@tc##1{\textcolor[rgb]{0.60,0.60,0.60}{##1}}}
\expandafter\def\csname PY@tok@na\endcsname{\def\PY@tc##1{\textcolor[rgb]{0.49,0.56,0.16}{##1}}}
\expandafter\def\csname PY@tok@nt\endcsname{\let\PY@bf=\textbf\def\PY@tc##1{\textcolor[rgb]{0.00,0.50,0.00}{##1}}}
\expandafter\def\csname PY@tok@nd\endcsname{\def\PY@tc##1{\textcolor[rgb]{0.67,0.13,1.00}{##1}}}
\expandafter\def\csname PY@tok@s\endcsname{\def\PY@tc##1{\textcolor[rgb]{0.73,0.13,0.13}{##1}}}
\expandafter\def\csname PY@tok@sd\endcsname{\let\PY@it=\textit\def\PY@tc##1{\textcolor[rgb]{0.73,0.13,0.13}{##1}}}
\expandafter\def\csname PY@tok@si\endcsname{\let\PY@bf=\textbf\def\PY@tc##1{\textcolor[rgb]{0.73,0.40,0.53}{##1}}}
\expandafter\def\csname PY@tok@se\endcsname{\let\PY@bf=\textbf\def\PY@tc##1{\textcolor[rgb]{0.73,0.40,0.13}{##1}}}
\expandafter\def\csname PY@tok@sr\endcsname{\def\PY@tc##1{\textcolor[rgb]{0.73,0.40,0.53}{##1}}}
\expandafter\def\csname PY@tok@ss\endcsname{\def\PY@tc##1{\textcolor[rgb]{0.10,0.09,0.49}{##1}}}
\expandafter\def\csname PY@tok@sx\endcsname{\def\PY@tc##1{\textcolor[rgb]{0.00,0.50,0.00}{##1}}}
\expandafter\def\csname PY@tok@m\endcsname{\def\PY@tc##1{\textcolor[rgb]{0.40,0.40,0.40}{##1}}}
\expandafter\def\csname PY@tok@gh\endcsname{\let\PY@bf=\textbf\def\PY@tc##1{\textcolor[rgb]{0.00,0.00,0.50}{##1}}}
\expandafter\def\csname PY@tok@gu\endcsname{\let\PY@bf=\textbf\def\PY@tc##1{\textcolor[rgb]{0.50,0.00,0.50}{##1}}}
\expandafter\def\csname PY@tok@gd\endcsname{\def\PY@tc##1{\textcolor[rgb]{0.63,0.00,0.00}{##1}}}
\expandafter\def\csname PY@tok@gi\endcsname{\def\PY@tc##1{\textcolor[rgb]{0.00,0.63,0.00}{##1}}}
\expandafter\def\csname PY@tok@gr\endcsname{\def\PY@tc##1{\textcolor[rgb]{1.00,0.00,0.00}{##1}}}
\expandafter\def\csname PY@tok@ge\endcsname{\let\PY@it=\textit}
\expandafter\def\csname PY@tok@gs\endcsname{\let\PY@bf=\textbf}
\expandafter\def\csname PY@tok@gp\endcsname{\let\PY@bf=\textbf\def\PY@tc##1{\textcolor[rgb]{0.00,0.00,0.50}{##1}}}
\expandafter\def\csname PY@tok@go\endcsname{\def\PY@tc##1{\textcolor[rgb]{0.53,0.53,0.53}{##1}}}
\expandafter\def\csname PY@tok@gt\endcsname{\def\PY@tc##1{\textcolor[rgb]{0.00,0.27,0.87}{##1}}}
\expandafter\def\csname PY@tok@err\endcsname{\def\PY@bc##1{\setlength{\fboxsep}{0pt}\fcolorbox[rgb]{1.00,0.00,0.00}{1,1,1}{\strut ##1}}}
\expandafter\def\csname PY@tok@kc\endcsname{\let\PY@bf=\textbf\def\PY@tc##1{\textcolor[rgb]{0.00,0.50,0.00}{##1}}}
\expandafter\def\csname PY@tok@kd\endcsname{\let\PY@bf=\textbf\def\PY@tc##1{\textcolor[rgb]{0.00,0.50,0.00}{##1}}}
\expandafter\def\csname PY@tok@kn\endcsname{\let\PY@bf=\textbf\def\PY@tc##1{\textcolor[rgb]{0.00,0.50,0.00}{##1}}}
\expandafter\def\csname PY@tok@kr\endcsname{\let\PY@bf=\textbf\def\PY@tc##1{\textcolor[rgb]{0.00,0.50,0.00}{##1}}}
\expandafter\def\csname PY@tok@bp\endcsname{\def\PY@tc##1{\textcolor[rgb]{0.00,0.50,0.00}{##1}}}
\expandafter\def\csname PY@tok@fm\endcsname{\def\PY@tc##1{\textcolor[rgb]{0.00,0.00,1.00}{##1}}}
\expandafter\def\csname PY@tok@vc\endcsname{\def\PY@tc##1{\textcolor[rgb]{0.10,0.09,0.49}{##1}}}
\expandafter\def\csname PY@tok@vg\endcsname{\def\PY@tc##1{\textcolor[rgb]{0.10,0.09,0.49}{##1}}}
\expandafter\def\csname PY@tok@vi\endcsname{\def\PY@tc##1{\textcolor[rgb]{0.10,0.09,0.49}{##1}}}
\expandafter\def\csname PY@tok@vm\endcsname{\def\PY@tc##1{\textcolor[rgb]{0.10,0.09,0.49}{##1}}}
\expandafter\def\csname PY@tok@sa\endcsname{\def\PY@tc##1{\textcolor[rgb]{0.73,0.13,0.13}{##1}}}
\expandafter\def\csname PY@tok@sb\endcsname{\def\PY@tc##1{\textcolor[rgb]{0.73,0.13,0.13}{##1}}}
\expandafter\def\csname PY@tok@sc\endcsname{\def\PY@tc##1{\textcolor[rgb]{0.73,0.13,0.13}{##1}}}
\expandafter\def\csname PY@tok@dl\endcsname{\def\PY@tc##1{\textcolor[rgb]{0.73,0.13,0.13}{##1}}}
\expandafter\def\csname PY@tok@s2\endcsname{\def\PY@tc##1{\textcolor[rgb]{0.73,0.13,0.13}{##1}}}
\expandafter\def\csname PY@tok@sh\endcsname{\def\PY@tc##1{\textcolor[rgb]{0.73,0.13,0.13}{##1}}}
\expandafter\def\csname PY@tok@s1\endcsname{\def\PY@tc##1{\textcolor[rgb]{0.73,0.13,0.13}{##1}}}
\expandafter\def\csname PY@tok@mb\endcsname{\def\PY@tc##1{\textcolor[rgb]{0.40,0.40,0.40}{##1}}}
\expandafter\def\csname PY@tok@mf\endcsname{\def\PY@tc##1{\textcolor[rgb]{0.40,0.40,0.40}{##1}}}
\expandafter\def\csname PY@tok@mh\endcsname{\def\PY@tc##1{\textcolor[rgb]{0.40,0.40,0.40}{##1}}}
\expandafter\def\csname PY@tok@mi\endcsname{\def\PY@tc##1{\textcolor[rgb]{0.40,0.40,0.40}{##1}}}
\expandafter\def\csname PY@tok@il\endcsname{\def\PY@tc##1{\textcolor[rgb]{0.40,0.40,0.40}{##1}}}
\expandafter\def\csname PY@tok@mo\endcsname{\def\PY@tc##1{\textcolor[rgb]{0.40,0.40,0.40}{##1}}}
\expandafter\def\csname PY@tok@ch\endcsname{\let\PY@it=\textit\def\PY@tc##1{\textcolor[rgb]{0.25,0.50,0.50}{##1}}}
\expandafter\def\csname PY@tok@cm\endcsname{\let\PY@it=\textit\def\PY@tc##1{\textcolor[rgb]{0.25,0.50,0.50}{##1}}}
\expandafter\def\csname PY@tok@cpf\endcsname{\let\PY@it=\textit\def\PY@tc##1{\textcolor[rgb]{0.25,0.50,0.50}{##1}}}
\expandafter\def\csname PY@tok@c1\endcsname{\let\PY@it=\textit\def\PY@tc##1{\textcolor[rgb]{0.25,0.50,0.50}{##1}}}
\expandafter\def\csname PY@tok@cs\endcsname{\let\PY@it=\textit\def\PY@tc##1{\textcolor[rgb]{0.25,0.50,0.50}{##1}}}

\def\PYZbs{\char`\\}
\def\PYZus{\char`\_}
\def\PYZob{\char`\{}
\def\PYZcb{\char`\}}
\def\PYZca{\char`\^}
\def\PYZam{\char`\&}
\def\PYZlt{\char`\<}
\def\PYZgt{\char`\>}
\def\PYZsh{\char`\#}
\def\PYZpc{\char`\%}
\def\PYZdl{\char`\$}
\def\PYZhy{\char`\-}
\def\PYZsq{\char`\'}
\def\PYZdq{\char`\"}
\def\PYZti{\char`\~}
% for compatibility with earlier versions
\def\PYZat{@}
\def\PYZlb{[}
\def\PYZrb{]}
\makeatother


    % For linebreaks inside Verbatim environment from package fancyvrb.
    \makeatletter
        \newbox\Wrappedcontinuationbox
        \newbox\Wrappedvisiblespacebox
        \newcommand*\Wrappedvisiblespace {\textcolor{red}{\textvisiblespace}}
        \newcommand*\Wrappedcontinuationsymbol {\textcolor{red}{\llap{\tiny$\m@th\hookrightarrow$}}}
        \newcommand*\Wrappedcontinuationindent {3ex }
        \newcommand*\Wrappedafterbreak {\kern\Wrappedcontinuationindent\copy\Wrappedcontinuationbox}
        % Take advantage of the already applied Pygments mark-up to insert
        % potential linebreaks for TeX processing.
        %        {, <, #, %, $, ' and ": go to next line.
        %        _, }, ^, &, >, - and ~: stay at end of broken line.
        % Use of \textquotesingle for straight quote.
        \newcommand*\Wrappedbreaksatspecials {%
            \def\PYGZus{\discretionary{\char`\_}{\Wrappedafterbreak}{\char`\_}}%
            \def\PYGZob{\discretionary{}{\Wrappedafterbreak\char`\{}{\char`\{}}%
            \def\PYGZcb{\discretionary{\char`\}}{\Wrappedafterbreak}{\char`\}}}%
            \def\PYGZca{\discretionary{\char`\^}{\Wrappedafterbreak}{\char`\^}}%
            \def\PYGZam{\discretionary{\char`\&}{\Wrappedafterbreak}{\char`\&}}%
            \def\PYGZlt{\discretionary{}{\Wrappedafterbreak\char`\<}{\char`\<}}%
            \def\PYGZgt{\discretionary{\char`\>}{\Wrappedafterbreak}{\char`\>}}%
            \def\PYGZsh{\discretionary{}{\Wrappedafterbreak\char`\#}{\char`\#}}%
            \def\PYGZpc{\discretionary{}{\Wrappedafterbreak\char`\%}{\char`\%}}%
            \def\PYGZdl{\discretionary{}{\Wrappedafterbreak\char`\$}{\char`\$}}%
            \def\PYGZhy{\discretionary{\char`\-}{\Wrappedafterbreak}{\char`\-}}%
            \def\PYGZsq{\discretionary{}{\Wrappedafterbreak\textquotesingle}{\textquotesingle}}%
            \def\PYGZdq{\discretionary{}{\Wrappedafterbreak\char`\"}{\char`\"}}%
            \def\PYGZti{\discretionary{\char`\~}{\Wrappedafterbreak}{\char`\~}}%
        }
        % Some characters . , ; ? ! / are not pygmentized.
        % This macro makes them "active" and they will insert potential linebreaks
        \newcommand*\Wrappedbreaksatpunct {%
            \lccode`\~`\.\lowercase{\def~}{\discretionary{\hbox{\char`\.}}{\Wrappedafterbreak}{\hbox{\char`\.}}}%
            \lccode`\~`\,\lowercase{\def~}{\discretionary{\hbox{\char`\,}}{\Wrappedafterbreak}{\hbox{\char`\,}}}%
            \lccode`\~`\;\lowercase{\def~}{\discretionary{\hbox{\char`\;}}{\Wrappedafterbreak}{\hbox{\char`\;}}}%
            \lccode`\~`\:\lowercase{\def~}{\discretionary{\hbox{\char`\:}}{\Wrappedafterbreak}{\hbox{\char`\:}}}%
            \lccode`\~`\?\lowercase{\def~}{\discretionary{\hbox{\char`\?}}{\Wrappedafterbreak}{\hbox{\char`\?}}}%
            \lccode`\~`\!\lowercase{\def~}{\discretionary{\hbox{\char`\!}}{\Wrappedafterbreak}{\hbox{\char`\!}}}%
            \lccode`\~`\/\lowercase{\def~}{\discretionary{\hbox{\char`\/}}{\Wrappedafterbreak}{\hbox{\char`\/}}}%
            \catcode`\.\active
            \catcode`\,\active
            \catcode`\;\active
            \catcode`\:\active
            \catcode`\?\active
            \catcode`\!\active
            \catcode`\/\active
            \lccode`\~`\~
        }
    \makeatother

    \let\OriginalVerbatim=\Verbatim
    \makeatletter
    \renewcommand{\Verbatim}[1][1]{%
        %\parskip\z@skip
        \sbox\Wrappedcontinuationbox {\Wrappedcontinuationsymbol}%
        \sbox\Wrappedvisiblespacebox {\FV@SetupFont\Wrappedvisiblespace}%
        \def\FancyVerbFormatLine ##1{\hsize\linewidth
            \vtop{\raggedright\hyphenpenalty\z@\exhyphenpenalty\z@
                \doublehyphendemerits\z@\finalhyphendemerits\z@
                \strut ##1\strut}%
        }%
        % If the linebreak is at a space, the latter will be displayed as visible
        % space at end of first line, and a continuation symbol starts next line.
        % Stretch/shrink are however usually zero for typewriter font.
        \def\FV@Space {%
            \nobreak\hskip\z@ plus\fontdimen3\font minus\fontdimen4\font
            \discretionary{\copy\Wrappedvisiblespacebox}{\Wrappedafterbreak}
            {\kern\fontdimen2\font}%
        }%

        % Allow breaks at special characters using \PYG... macros.
        \Wrappedbreaksatspecials
        % Breaks at punctuation characters . , ; ? ! and / need catcode=\active
        \OriginalVerbatim[#1,codes*=\Wrappedbreaksatpunct]%
    }
    \makeatother

    % Exact colors from NB
    \definecolor{incolor}{HTML}{303F9F}
    \definecolor{outcolor}{HTML}{D84315}
    \definecolor{cellborder}{HTML}{CFCFCF}
    \definecolor{cellbackground}{HTML}{F7F7F7}

    % prompt
    \makeatletter
    \newcommand{\boxspacing}{\kern\kvtcb@left@rule\kern\kvtcb@boxsep}
    \makeatother
    \newcommand{\prompt}[4]{
        {\ttfamily\llap{{\color{#2}[#3]:\hspace{3pt}#4}}\vspace{-\baselineskip}}
    }



    % Prevent overflowing lines due to hard-to-break entities
    \sloppy
    % Setup hyperref package
    \hypersetup{
      breaklinks=true,  % so long urls are correctly broken across lines
      colorlinks=true,
      urlcolor=urlcolor,
      linkcolor=linkcolor,
      citecolor=citecolor,
      }
    % Slightly bigger margins than the latex defaults

    \geometry{verbose,tmargin=1in,bmargin=1in,lmargin=1in,rmargin=1in}



\renewcommand{\PY}[2]{{#2}}
\usepackage{fancyhdr}
\pagestyle{fancy}
\rhead{\color{gray}\sf\small\rightmark}
\lhead{\nouppercase{\color{gray}\sf\small\leftmark}}
\cfoot{\color{gray}\sf\thepage}
\renewcommand{\footrulewidth}{1pt}
\begin{document}





    \hypertarget{solutions-to-unix-for-bioinformatics}{%
\section{Solutions to Unix for
Bioinformatics}\label{solutions-to-unix-for-bioinformatics}}

\hypertarget{introduction}{%
\subsection{Introduction}\label{introduction}}

No questions in this section.

\hypertarget{basic-unix}{%
\subsection{Basic Unix}\label{basic-unix}}

\textbf{1.} \texttt{ls\ -al}\\
\textbf{2.} There are 4 files in the directory (and 2 subdirectories).
You can use \texttt{ls\ -l} to look inside the directory. This will show
you which of the contents are files and which are directories. Don't
forget to also include the \texttt{-a} option to show any hidden files:

\texttt{ls\ -la\ Pfalciparum}

\textbf{3.} Malaria.fa is the largest file. You can add the \texttt{-h}
option to the command above to make the size of the files more readable.

\textbf{4.} \texttt{cd\ Pfalciparum}

\textbf{5.} There is one file in the fasta directory (hint: it's a
hidden file!).

\textbf{6.} \texttt{cp\ Pfalciparum.bed\ annotation}

\textbf{7.} \texttt{mv\ *.fa\ fasta}

\textbf{8.} 4 files.

\textbf{9.} There are 6 GFFs in the unix directory. To search from the
Unix directory, you can either use \texttt{cd} to move up to the
directory, or you can specify the path in the \texttt{find} command.
This can either be the absolute path, which you can get from
\texttt{pwd}, or you can use the relative path, like so:

\texttt{find\ ../../..\ -name\ *.gff}

\textbf{10.} There are 7 fasta files in the unix directory. Note that
fasta files normally end with \texttt{.fa} OR \texttt{.fasta}, so wee
need to make sure we look for both of these, by adding a wildcard after
\texttt{fa}:

\texttt{find\ ../../..\ -name\ *.fa*}

    \hypertarget{looking-inside-files}{%
\subsection{Looking inside files}\label{looking-inside-files}}

\textbf{1.}
\texttt{head\ -n\ 500\ Styphi.gff\ \textgreater{}\ Styphi.500.gff}

\textbf{2.} There are 6213 lines in the file. Use the \texttt{-l}
option:

\texttt{wc\ -l\ Pfalciparum.bed}

\textbf{3.} \texttt{sort\ -k\ 1\ -k\ 2\ -n\ Pfalciparum.bed}

\textbf{4.} Here is one way to do this: First use \texttt{awk} to get
the first column of the file. Sort this and then use the \texttt{-c}
option for \texttt{uniq} to count how many entries each chromosome has:

\texttt{awk\ \textquotesingle{}\{\ print\ \$1\ \}\textquotesingle{}\ Pfalciparum.bed\ \textbar{}\ sort\ \textbar{}\ uniq\ -c}

The expected output would be:

\begin{verbatim}
 190 01
 264 02
 287 03
 292 04
 357 05
 373 06
 395 07
 374 08
 425 09
 452 10
 553 11
 621 12
 773 13
 857 14
\end{verbatim}

    \hypertarget{searching-inside-files-with-grep}{%
\subsection{\texorpdfstring{Searching inside files with
\texttt{grep}}{Searching inside files with grep}}\label{searching-inside-files-with-grep}}

\textbf{1.} \texttt{grep\ "\^{}\textgreater{}"\ exercises.fasta}

\textbf{2.} There are 1000 sequences. We can use -c to count the number
of matches:

\texttt{grep\ -c\ "\^{}\textgreater{}"\ exercises.fasta}

Or pipe into wc:

\texttt{grep\ "\^{}\textgreater{}"\ exercises.fasta\ \textbar{}\ wc\ -l}

\textbf{3.} Yes, three of them:

\begin{verbatim}
>sequence27 spaces in the name
>sequence52 another with spaces
>sequence412 yet another with spaces
\end{verbatim}

One option is two greps piped together:

\texttt{grep\ "\^{}\textgreater{}"\ exercises.fasta\ \textbar{}\ grep\ "\ "}

Alternatively, in one regular expression

\texttt{grep\ "\^{}\textgreater{}.*\ .*"\ exercises.fasta}

\textbf{4.} \texttt{grep\ -v\ "\^{}\textgreater{}"\ exercises.fasta}

\textbf{5.} Three. First extract the sequences, then search for n:

\texttt{grep\ -v\ "\^{}\textgreater{}"\ exercises.fasta\ \textbar{}\ grep\ -c\ -i\ n}

\textbf{6.} Yes, one sequence. Try:

\texttt{grep\ -v\ "\^{}\textgreater{}"\ exercises.fasta\ \textbar{}\ grep\ -i\ -v\ "\^{}{[}acgtn{]}.*\$"}

Alternatively, we can use the \^{} to ask for matches NOT in the
alphabet {[}acgtn{]}

\texttt{grep\ -v\ "\^{}\textgreater{}"\ exercises.fasta\ \textbar{}\ grep\ -i\ "{[}\^{}ACGTN{]}"}

\textbf{7.} 66 sequences. Try:

\texttt{grep\ -v\ "\^{}\textgreater{}"\ exercises.fasta\ \textbar{}\ grep\ -c\ "GC{[}AT{]}GC"}

\textbf{8.} We found the total number of sequences earlier:

\texttt{grep\ -c\ "\^{}\textgreater{}"\ exercises.fasta}

\ldots{} which outputs 1000

This finds the number of unique names:

\texttt{grep\ "\^{}\textgreater{}"\ exercises.fasta\ \textbar{}\ sort\ \textbar{}\ uniq\ \textbar{}\ wc\ -l}

\ldots{} which outputs 999.

Therefore there is 1000 - 999 = 1 name repeated.

    \hypertarget{file-processing-with-awk}{%
\subsection{\texorpdfstring{File processing with
\texttt{awk}}{File processing with awk}}\label{file-processing-with-awk}}

\textbf{1.} Using:

\texttt{awk\ -F"\textbackslash{}t"\ \textquotesingle{}\{print\ \$1\}\textquotesingle{}\ exercises.bed\ \textbar{}\ sort\ -u}

Should give you:

\begin{verbatim}
contig-1
contig-3
contig-4
contig-5
scaffold-2
\end{verbatim}

\textbf{2.} There are 5 contigs. Use the command from the previous
exercise and count the number of lines with \texttt{wc}:

\texttt{awk\ -F"\textbackslash{}t"\ \textquotesingle{}\{print\ \$1\}\textquotesingle{}\ exercises.bed\ \textbar{}\ sort\ -u\ \textbar{}\ wc\ -l}

\textbf{3.} There are 164 features on the positive strand. Try:

\texttt{awk\ -F"\textbackslash{}t"\ \textquotesingle{}\$6=="+"\textquotesingle{}\ exercises.bed\ \textbar{}\ wc\ -l}

\textbf{4.} Ther are 124 features on the negative strand. Try:

\texttt{awk\ -F"\textbackslash{}t"\ \textquotesingle{}\$6=="-"\textquotesingle{}\ exercises.bed\ \textbar{}\ wc\ -l}

\textbf{5.} There are 293 genes. Try:

\texttt{awk\ -F"\textbackslash{}t"\ \textquotesingle{}\$4\ \textasciitilde{}\ /gene/\textquotesingle{}\ exercises.bed\ \textbar{}\ wc\ -l}

\textbf{6.} 5 genes have no strand assigned to them. Try:

\texttt{awk\ -F"\textbackslash{}t"\ \textquotesingle{}\$4\ \textasciitilde{}\ /gene/\ \&\&\ \$6\ !=\ "-"\ \&\&\ \$6\ !=\ "+"\textquotesingle{}\ exercises.bed\ \textbar{}\ wc\ -l}

\textbf{7.} Yes (6 of them are). First, the number of genes was found
earlier:

\texttt{awk\ -F"\textbackslash{}t"\ \textquotesingle{}\$4\ \textasciitilde{}\ /gene/\textquotesingle{}\ exercises.bed\ \textbar{}\ wc\ -l}

The number of unique names is:

\texttt{awk\ -F"\textbackslash{}t"\ \textquotesingle{}\$4\ \textasciitilde{}\ /gene/\ \{print\ \$4\}\textquotesingle{}\ exercises.bed\ \textbar{}\ sort\ -u\ \textbar{}\ wc\ -l}

Alternatively, the names can be found like this:

\texttt{awk\ -F"\textbackslash{}t"\ \textquotesingle{}\$4\ \textasciitilde{}\ /gene/\ \{print\ \$4\}\textquotesingle{}\ exercises.bed\ \ \textbar{}\ sort\ \textbar{}\ uniq\ -c\ \textbar{}\ awk\ \textquotesingle{}\$1\textgreater{}1\textquotesingle{}}

\textbf{9.} 18541. Try:

\texttt{awk\ -F"\textbackslash{}t"\ \textquotesingle{}\$4=="repeat"\ \{score+=\$5\}\ END\ \{print\ score\}\textquotesingle{}\ exercises.bed}

\textbf{10.} There are 75 features in contig-1. Try:

\texttt{awk\ -F"\textbackslash{}t"\ \textquotesingle{}\$1\ ==\ "contig-1"\textquotesingle{}\ exercises.bed\ \ \textbar{}\ wc\ -l}

\textbf{11.} There are 12 repeats in contig-1. Try:

\texttt{awk\ -F"\textbackslash{}t"\ \textquotesingle{}\$1\ ==\ "contig-1"\ \&\&\ \$4\ ==\ "repeat"\textquotesingle{}\ exercises.bed\ \ \textbar{}\ wc\ -l}

\textbf{12.} The mean score is 560.833. Try:

\texttt{awk\ -F"\textbackslash{}t"\ \textquotesingle{}\$1\ ==\ "contig-1"\ \&\&\ \$4\ ==\ "repeat"\ \{score+=\$5;\ count++\}\ END\{print\ score/count\}\textquotesingle{}\ exercises.bed}

    \hypertarget{bash-scripts}{%
\subsection{BASH scripts}\label{bash-scripts}}

No questions in this section.

\hypertarget{advanced-bash}{%
\subsection{Advanced BASH}\label{advanced-bash}}

\textbf{1.} Here is an example of what this script could look like:

\begin{verbatim}
#!/usr/bin/env bash
set -e

# check that the correct number of options was given.
# If not, then write a message explaining how to use the
# script, and then exit.
if [ $# -ne 1 ]
then
    echo "usage: example_1.sh filename"
    echo
    echo "Prints the number of lines in the file"
    exit
fi

# Use sensibly named variable
filename=$1

# check if the input file exists
if [ ! -f $filename ]
then
    echo "File '$filename' not found! Cannot continue"
    exit
fi


# If still here, we can count the number of lines
number_of_lines=$(wc -l $filename | awk '{print $1}')
echo "There are $number_of_lines lines in the file $filename"
\end{verbatim}

\textbf{2.} Here is an example of what this script could look like:

\begin{verbatim}
#!/usr/bin/env bash
set -e
for filename in ../scripts/loop_files/*; do ./exercise_1.sh $filename; done
\end{verbatim}

\textbf{3.} Here is an example of what this script could look like:

\begin{verbatim}
#!/usr/bin/env bash
set -e

# Check if the right number of options given.
# If not, print the usage
if [ $# -ne 1 ]
    then
        echo "usage: example_3.sh in.gff"
        echo
        echo "Gathers some summary information from a gff file"
        exit
fi

# store the filename in a better named variable
infile=$1


# Stop if the input file does not exist
if [ ! -f $infile ]
then
    echo "File '$infile' not found! Cannot continue"
    exit 1
fi

echo "Gathering data for $infile..."


# Gather various stats on the file...


# Total number of lines/records in file
total_records=$(wc -l $infile | awk '{print $1}')
echo "File has $total_records records in total"


# Get the sources from column 2.
echo
echo "The sources in the file are:"
awk '{print $2}' $infile | sort -u


# Count the sources
echo
echo "Count of sources, sorted by most common"
awk '{print $2}' $infile | sort | uniq -c | sort -n


# Count which features have no score
echo
echo "Count of features that have no score"
awk '$6=="." {print $3}' $infile | sort | uniq -c


# Find how many bad coords there are
echo
bad_coords=$(awk '$5 < $4' $infile | wc -l | awk '{print $1}')
echo "Records with bad coordinates: $bad_coords"



#_______________________________________________________________#
#                                                               #
#      WARNING: the following examples are more advanced!       #
#_______________________________________________________________#

# if there were records with bad coords, find the sources responsible
if [ $bad_coords != 0 ]
then
    echo
    echo "Sources of bad coordinates:"
    # Instead getting one source per line, pipe into awk again to print them
    # on one line with semicolon and space between the names
    awk '$5 < $4 {print $2}' $infile | sort -u | awk '{sources=sources"; "$1} END{print substr(sources, 3)}'
fi



# Count of the features. Instead of using awk .... |sort | uniq -c
# we will just use awk. Compare this with the above method
# used to count the sources. Although it is a longer command, it is more efficient
echo
echo "Count of each feature:"
awk '{counts[$3]++} END{for (feature in counts){print feature"\t"counts[feature]}}' $infile | sort -k2n


# This example is even more complicated! It uses a loop to
# get the mean score of the genes, broken down by source.
echo
echo "Getting mean scores for each source..."
for source in `awk '{print $2}' $infile | sort | uniq`
do
    awk -v s=$source '$2==s {total+=$6; count++} END{print "Mean score for", s":\t", total/count}' $infile
done


# We can use awk to split the input into multiple output files.
# Writing print "line" > filename will append the string "line"
# to a file called filename. If a file called filename
# does not exist already, then it will be created.
#
# Write a new gff for each of the sources in the original input gff file
echo
echo "Writing a file per source of the original gff file $infile to files called split.*"
awk '{filename="split."$2".gff"; print $0 > filename}' $infile
echo " ... done!"
\end{verbatim}


    % Add a bibliography block to the postdoc



\end{document}
